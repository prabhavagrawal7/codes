
\documentclass{article}

\usepackage{lipsum} % for dummy text

\title{My Document}
\author{John Doe}
\date{\today}

\begin{document}

\maketitle

Given an array $a_1, a_2, \ldots, a_n$ such that $a_i = i$ for all $i$. For an integer $k \geq 2$, the operation $Swap(k)$ is defined as:

Let $d$ be the largest divisor of $k$ which is not equal to $k$. Swap the elements $a_d$ and $a_k$.

Suppose you perform $Swap(i)$ for all $2 \leq i \leq n$ in order. Find the position of 1 in the resulting array. In other words, find such $j$ that $a_j = 1$ after performing these operations.

Input

Each test case contains multiple test cases. The first line contains the number of test cases $t$ ($1 \leq t \leq 10^4$). The description of the test cases follows.

The only line of each test case contains one integer $n$ ($1 \leq n \leq 10^9$) — the length of the array $a$.

Output

For each test case, output the position of 1 in the resulting array.

Example

input

4

1

4

5

120240229

output

1

4

4

67108864

Note

In the first test case, the array is [1] and there are no operations performed.

In the second test case, $a$ changes as follows:

Initially, $a$ is [1,2,3,4].

After performing $Swap(2)$, $a$ changes to [2,\underline{1},3,4].

After performing $Swap(3)$, $a$ changes to [3,1,2,4].

After performing $Swap(4)$, $a$ changes to [3,4,1,2] (that is, the element 1 lies on index 4). Thus, the answer is 4.

\end{document}


